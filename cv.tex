\documentclass[11pt,a4paper,sans]{moderncv}        % possible options include font size ('10pt', '11pt' and '12pt'), paper size ('a4paper', 'letterpaper', 'a5paper', 'legalpaper', 'executivepaper' and 'landscape') and font family ('sans' and 'roman')

% moderncv themes
\moderncvstyle{classic}                             % style options are 'casual' (default), 'classic', 'banking', 'oldstyle' and 'fancy'
\moderncvcolor{blue}                               % color options 'black', 'blue' (default), 'burgundy', 'green', 'grey', 'orange', 'purple' and 'red'
%\renewcommand{\familydefault}{\sfdefault}         % to set the default font; use '\sfdefault' for the default sans serif font, '\rmdefault' for the default roman one, or any tex font name
%\nopagenumbers{}                                  % uncomment to suppress automatic page numbering for CVs longer than one page

% character encoding
\usepackage[utf8]{inputenc}                       % if you are not using xelatex ou lualatex, replace by the encoding you are using
%\usepackage{CJKutf8}                              % if you need to use CJK to typeset your resume in Chinese, Japanese or Korean

\usepackage{url}
\usepackage{hyperref}

% adjust the page margins
\usepackage[scale=0.90]{geometry}
%\setlength{\hintscolumnwidth}{3cm}                % if you want to change the width of the column with the dates
%\setlength{\makecvtitlenamewidth}{10cm}           % for the 'classic' style, if you want to force the width allocated to your name and avoid line breaks. be careful though, the length is normally calculated to avoid any overlap with your personal info; use this at your own typographical risks...

% personal data
\name{}{K.N Anantha nandanan}
\address{Kollam, Kerala, India}% optional, remove / comment the line if not wanted; the "postcode city" and "country" arguments can be omitted or provided empty
\phone[mobile]{+91 8921397828}                   % optional, remove / comment the line if not wanted; the optional "type" of the phone can be "mobile" (default), "fixed" or "fax"
\email{ananthanandanan@gmail.com}                               % optional, remove / comment the line if not wanted
\homepage{ananthanandanan.github.io}                         % optional, remove / comment the line if not wanted
\social[linkedin]{DE-BLANK}                        % optional, remove / comment the line if not wanted

% bibliography adjustements (only useful if you make citations in your resume, or print a list of publications using BibTeX)
%   to show numerical labels in the bibliography (default is to show no labels)
\makeatletter\renewcommand*{\bibliographyitemlabel}{\@biblabel{\arabic{enumiv}}}\makeatother
%   to redefine the bibliography heading string ("Publications")
%\renewcommand{\refname}{Articles}

% bibliography with mutiple entries
%\usepackage{multibib}
%\newcites{book,misc}{{Books},{Others}}
%----------------------------------------------------------------------------------
%            content
%----------------------------------------------------------------------------------
\begin{document}
%\begin{CJK*}{UTF8}{gbsn}                          % to typeset your resume in Chinese using CJK
%-----       resume       ---------------------------------------------------------
\makecvtitle

\section{OBJECTIVE}
\cvitem{}{Seeking a position in the field of Computer Science where I can utilize my skills to further work towards personal and professional development and contribute towards the prosperity of the organization. Highly motivated and eager to learn new things.}

\section{EDUCATION}
\cventry{2019-2023 Ongoing}{B.Tech in Computer Science and Engineering}{Amrita Vishwa Vidyapeetham}{Kollam, Kerala, India}{}
{\textit{CGPA: 8.7/10}}{}  % arguments 3 to 6 can be left empty
\cventry{2018}{City Central School}{}{Kollam, kerala, India}{}
{\textit{Percentage: 89.7\%  }}{}  % arguments 3 to 6 can be left empty
\cventry{2016}{City Central School}{}{Kollam, kerala, India}{}
{\textit{Percentage: 10CGPA  }}{}  % arguments 3 to 6 can be left empty

\section{EXPERIENCE}

\cventry{August 2020 - Ongoing}{Developer Student Club}{Member, Amrita School of Engineering}{}{}
{Developer Student Club is the university-based community groups for students interested in Google developer technologies.DSC helps us to grow our knowledge in a peer-to-peer learning environment and build solutions for local businesses and their community.}


\cventry{August 2019}{Member at amFOSS}{}{}{}
{amFOSS is the Free and Open Source Software club of my college. I have been an active member of the community from the time I joined college. I actively take part in all events and also help in organizing events hosted by amFOSS.}

\cventry{January 2020}{ Open source contributor at Oppia}{}{}{}
{ An open source contributor for the Angular based web-application developed by Oppia. Project Link: {\newline} 
\url{https://github.com/oppia/oppia}}{}  % arguments 3 to 6 can be left empty

\section{TECHNICAL PROJECTS}



\cventry{October 2020}{NAV-FT}{}{}{}
{This is a android flutter application which contains some of the features that we can use to speed up the process of vehicle fitness testing. New vehicle registration, New vehicle registration using image, check whether the inspection for a certain vehicle is pending or completed, Checks for the tyre condition etc. Tech stack - Android SDK, Flutter, Dart, Google API, Firebase, Docker, Tensorflow, AWS sageMaker. Project Link: {\newline}
\url{https://github.com/Fireboltz/NAV-FT}}{}

\cventry{April 2020}{NaWaB bot}{}{}{}
{This bot is a content curator for the topics related to the networks. It is the best solution to be updated in the latest technologies, developments and discussion of networks on twitter.When the script is running all the tweets related to networks are scraped and the tweet id with timestamp are stored in a .csv file. All the result and error logs are stored in result.log and error.log respectively. When the user sends a /start command to the bot, it starts sending the latest tweets. Tech stack used- Python, Twitter API, Telegram API. Project Link: {\newline}
\url{https://github.com/Team-SYNACKd/NaWaB}}{}

\cventry{October 2020}{Craig Scraper}{}{}{}
{Craig Scraper is a clone of Craiglist website built using Django. This web application provides a better user experience than the original Craiglist website. The user can input the search query and surf for commodities to purchase.  Tech stack - Django, beautifulSoup, Python, materialize css. Project Link: {\newline}
\url{https://github.com/ananthanandanan/Craig-Scraper}}{}


\cventry{October 2020}{CheckList}{}{}{}
{A general-purpose todo list application build using react. It helps the user to keep track of his day-to-day work, general activities. Tech stack - React.js, Material-UI, Firebase. Project Link: {\newline}
\url{https://github.com/ananthanandanan/Checklist}}{}


\cventry{August 2020}{Lorax}{}{}{}
{This is a android flutter application which helps people to be motivated to grow plants and trees. The user could track the progress of the growth of the tree, set reminders. Also post his progress in social media platforms to showcase his efforts and hence motivate others as well. Tech stack - Android SDK, Flutter, Dart, moor database, Google API, Firebase, Firebase cloud-Store. Project Link: {\newline}
\url{https://github.com/ashwinkey04/lorax}}{}

\cventry{January 2020}{Bibliungo}{}{}{}
{Bibliungo is a platform for avid readers in a locality to share and exchange books with each other. The app scans for other users in your vicinity who are searching for second hand books to buy or exchange. On finding a match, you will be given their contact information after which you may meet up IRL for your new book and hopefully a new friend!. Tech stack used - Java, Android SDK, Firebase, Google API.
Project Link : {\newline}
\url{https://github.com/AwesomeFruitSalad/Bibliungo}}

\cventry{January 2020 - May 2020}{Oppia tool - Oppia}{}{}{}
{Oppia is an online learning tool that enables anyone to easily create and share interactive activities (called 'explorations'). These activities simulate a one-on-one conversation with a tutor, making it possible for students to learn by doing while getting feedback. Tech stack used - Angular, Python, Google App Engine. Project Link: {\newline}
\url{https://github.com/oppia/oppia}}{}

\cventry{November 2019}{NY Weather}{}{}{}
{This is simple web application were the user could search for the current weather of a city of the users desire. Tech stack used - Django, Python, Open weather API. Project Link: {\newline}
\url{https://github.com/ananthanandanan/NY-Weather}}{}



\section{VOLUNTEER}
\cventry{October 2020}{Hacktoberfest-2020}{}{}{}
{Helped in volunteering and hosting hacktoberfest like last year where various workshops were taken on Git/GitHub, Introduction to Linux and open source.}
\cventry{October 2019}{Hacktoberfest2019}{}{}{}
{Helped in volunteering  hacktoberfest last year where various workshops were taken on GitHub, Gnome etc. }


\section{ACHIEVEMENTS}
\cventry{January 2020}{Hackverse}{}{}{}
{Was invited for conference of Digital Ocean, Elastic and Matic based on dockers, block chain, and elastic search services. }
\cventry{October 2020}{IEEE GovTechThon'20}{}{}{}
{Finished as a \textbf{runner up in the IEEE GovTechThon'20}, out of the 100+ teams selected for the hackathon. }



\section{Coursework}

%\renewcommand{\listitemsymbol}{-~} % Changes the symbol used for lists

\cvitem{Core Courses}{Data Structure, Object Oriented Programming, Elements of computing systems, Database Management}{}
\cvitem{Lab Courses}{Data Structure Lab, Object Oriented Programming, JAVA Lab, Digital Manufacturing}{}

\section{LANGUAGES}
\cvitemwithcomment{English}{Full Professional Proficiency}{}
\section{COMPUTER SKILLS}
\cvitem{OS}{Linux, Windows}
\cvitem{Programming Languages}{Python, Java, Javascript, C++, C, MySQL, Postgresql }
\cvitem{VCS}{Git}
\cvitem{Other Skills}{Android Development, Web Development, AutoCad,  Problem Solving, Tinkercad, Frama-c}

\section{INTERESTS}
\cvitem{Technical}{Networking, Cyber-Security, Automation,Data-analysis, Contributing to Open Source}
\cvitem{Hobbies}{Reading, Workout, solve puzzles }

\section{PERSONAL DETAILS}
\cvitem{DOB}{20th November, 2000}
\cvitem{Current Residence}{Kollam,Kerala, India}
\cvitem{Status}{Student}

\clearpage
%\clearpage\end{CJK*}                              % if you are typesetting your resume in Chinese using CJK; the \clearpage is required for fancyhdr to work correctly with CJK, though it kills the page numbering by making \lastpage undefined
\end{document}


%% end of file `template.tex'.